% \VignetteEngine{knitr::knitr}
% \VignetteEncoding{UTF-8}
% \VignetteIndexEntry{dtComb}

\documentclass[10pt]{article}
\usepackage[left=3cm, top=2.5cm, right=2.5cm, bottom=2cm]{geometry}
\usepackage[utf8]{inputenc}
\usepackage[colorlinks=true,linkcolor=blue,citecolor=blue,urlcolor=blue]{hyperref}
\usepackage{hyperref}
\hypersetup{
  colorlinks=true,
  linkcolor=black,
  filecolor=magenta,      
  urlcolor=blue
}

\usepackage{amsmath}
\usepackage{amsfonts}
\usepackage{amssymb}
\usepackage[numbers]{natbib}
\usepackage{nameref} 
\usepackage{booktabs}
\usepackage{caption}
\RequirePackage{graphicx, fancyvrb, textcomp}

\usepackage{authblk}
\setcounter{Maxaffil}{0}
\renewcommand\Affilfont{\itshape\small}

\newcommand{\dtComb}{\textit{dtComb}}
\newcommand{\CRANpkg}[1]{\href{https://cran.r-project.org/web/packages/#1/index.html}{\texttt{#1}}}
\newcommand{\Rfunction}[1]{\texttt{#1}}
\newcommand{\Rcode}[1]{\texttt{#1}}
\newcommand{\Rclass}[1]{\texttt{#1}}
\newcommand{\software}[1]{\texttt{#1}}



\title{\textbf{dtComb: A Comprehensive R Package for Combining Two Diagnostic Tests}}

\author[1,2]{S. İlayda YERLİTAŞ TAŞTAN}
\author[2]{Serra Berşan GENGEÇ}
\author[3]{Necla KOÇHAN}
\author[1,2]{Gözde ERTÜRK ZARARSIZ}
\author[4]{Selçuk KORKMAZ}
\author[1,2${}^{\dagger}$]{Gökmen ZARARSIZ}


\affil[1]{Erciyes University Faculty of Medicine Department of Biostatistics, Kayseri, Türkiye \vspace*{0.3em}}
\affil[2]{Erciyes University Drug Application and Research Center (ERFARMA) , Kayseri, Türkiye \vspace*{0.3em}}
\affil[3]{Izmir University of Economics-Department of Mathematics, 35330, İzmir, Türkiye \vspace*{0.3em}}
\affil[4]{Trakya University Faculty of Medicine Department of Biostatistics, Edirne, Türkiye \vspace*{0.3em}}
% 
\renewcommand\Authands{ and }

\date{
  \today
}
\usepackage{Sweave}
\begin{document}
\maketitle
\vspace*{10pt}

\begin{abstract}
\CRANpkg{dtComb} is a comprehensive R package that combines two different diagnostic tests. Using its extensive collection of 143 combination methods, the \CRANpkg{dtComb} package enables researchers to standardize their data and merge diagnostic tests. Users can load the dataset containing the reference list and the diagnostic tests they intend to utilize. The package includes combination methods grouped into four main categories: linear combination methods (\Rfunction{linComb}), non-linear combination methods (\Rfunction{nonlinComb}), mathematical operators (\Rfunction{mathComb}), and machine-learning algorithms (\Rfunction{mlComb}). The package incorporates eight specific combination methods from the literature within the scope of linear combination methods. Non-linear combination methods encompass statistical approaches like polynomial regression, penalized regression methods, and splines, incorporating the interactions between the diagnostic tests. Mathematical operators involve arithmetic operations and eight \texttt{distance measures} adaptable to various data structures. Finally, machine-learning algorithms include 113 models from the \CRANpkg{caret} package tailored for \CRANpkg{dtComb}'s data structure. The data standardization step includes five different methods: Z-score, T-score, Mean, Deviance, and Range standardization. The \CRANpkg{dtComb} integrates machine-learning approaches, enabling the utilization of preprocessing methods available in the \CRANpkg{caret} package for standardization purposes within the \CRANpkg{dtComb} environment. The \CRANpkg{dtComb} package allows users to fine-tune hyperparameters while building a model. This is accomplished through resampling techniques such as 10-fold cross-validation, bootstrapping, and 10-fold repeated cross-validation. Since machine-learning algorithms are directly adapted from the \CRANpkg{caret} package, all resampling methods available in the \CRANpkg{caret} package are applicable within the \CRANpkg{dtComb} environment. Following the model building, the \Rfunction{predict} function predicts the class labels and returns the combination scores of new observations from the test set. The \CRANpkg{dtComb} package is designed to be user-friendly and easy to use and is currently the most comprehensive package developed to combine diagnostic tests in the literature. This vignette was created to guide researchers on how to use this package. 
\vspace{1em}
\noindent\textbf{dtComb version:} 1.0.5
\end{abstract}
\section{Introduction}
Diagnostic tests are critical in distinguishing diseases and determining accurate diagnoses for patients, and they significantly impact clinical decisions. Beyond their fundamental role in medical diagnosis, these tests also aid in developing appropriate treatment strategies while lowering treatment costs. The widespread availability of these diagnostic tests depends on their accuracy, performance, and reliability. When it comes to diagnosing medical conditions, there may be several tests available, and some may perform better than others and eventually replace established protocols. Studies have shown that using multiple tests rather than relying on a single test improves diagnostic performance \citep{amini2019application, yu2019covariate, aznar2021incorporating}. A number of approaches to combining diagnostic tests are available in the literature. The \dtComb{} package includes a variety of combination methods existing in the literature, data standardization approaches for different data structure, and resampling methods for model building. 
In this vignette, users will learn how to combine two diagnostic tests with different combination methods. \dtComb{} package can be loaded as below: 
\begin{Schunk}
\begin{Sinput}
> library(dtComb)
\end{Sinput}
\end{Schunk}
\section{Preparing the input data}
The methods provided within this package are designed to require a \Rclass{DataFrame} comprising three columns, where the first column represents class labels, and the subsequent columns correspond to the values of the corresponding markers. The class label is a binary variable (i.e., negative/positive, present/absent) representing the outcomes of a reference test used in disease precision. 
This vignette will use the dataset \texttt{laparotomy}, included in this package. This dataset contains information from patients admitted to the General Surgery Department of Erciyes University Medical Faculty with complaints regarding abdominal pain. The dataset comprises 225 patients split into two groups: those requiring immediate laparotomy (110 patients) and those not requiring it (115 patients). Patients who had surgery due to postoperative pathologies are in the first group, whereas those with negative laparotomy results belong to the second group \citep{akyildiz2010value}. 
\begin{Schunk}
\begin{Sinput}
> data(laparotomy)
> head(laparotomy)